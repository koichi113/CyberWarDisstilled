\documentclass[../th_cyber_warfare_distilled.tex]{subfiles}
 
\begin{document}
\chapter{เครื่องมือและเทคนิคที่เกี่ยวข้องกับปฏิบัติการไซเบอร์}
\label{chapter:cyberwar_tools}

\section{เครื่องมือเชิงรุก (Offensive Tools)}
\subsection{มัลแวร์ประเภต่างๆ (Malware)} หมายถึงซอฟต์แวร์ที่มีชุดคำสั่งที่สามารถสำเนาตัวเองเข้าไปในระบบคอมพิวเตอร์แล้วสามารถแพร่กระจายไปสู่ระบบคอมพิวเตอร์อื่นๆผ่าน เครือข่ายคอมพิวเตอร์ ระบบจดหมายอิเล็กทรอนิกส์ หรืออุปกรณ์ต่อพ่วงต่างๆเช่น USB Drive โดยซอฟต์แวร์เหล่านั้นถูกพัฒนาขึ้นเพื่อละเมิดความมั่นคงปลอดภัยของทรัพยากรสารสนเทศและการสื่อสารจึงส่งผลต่อ การรักษาความลับ การรักษาความครบถ้วนสมบูรณ์ และการรักษาความพร้อมใช้ของทรัพยากรที่มัลแวร์นั้นทำงานอยู่ สามารถจำแนกประเภทตามคุณลักษณะจำเพาะของมัลแร์ได้หลายประเภทเช่น 
\begin{itemize}
	\item ไวรัสคอมพิวเตอร์ (Computer Virus) หมายถึงมัลแวร์ที่มีความสามารถในการติดเข้าระบบคอมพิวเตอร์แต่ไม่สามารถแพร่กระจายตัวเองผ่านระบบเครือข่ายได้ และจะต้องใช้สื่ออื่นในการแพร่กระจายเช่นดิสก์ หรือ USB Drive เป็นต้น
	\item หนอนอินเทอร์เน็ต (Internet Worm) หมายถึงมัลแวร์ที่เผยแพร่ตัวเองผ่านระบบเครือข่ายคอมพิวเตอร์ โดยอาจมีความสามารถที่หลากหลายตั้งแต่การส่งจดหมายอิเล็กทรอนิกส์ไปให้ผู้อื่น 
	\item คีย์ล๊อกเกอร์ (Key Logger) หมายถึงมัลแวร์ที่เมื่อถูกติดตั้งบนเครื่องคอมพิวเตอร์แล้วจะทำการดักรับข้อมูลการกดคีย์บอร์ดของผู้ใช้งานแล้วส่งต่อข้อมูลดังกล่าวไปให้ผู้ไม่ประสงค์ดี เป็นต้น
	\item โทรจัน (Trojan) หมายถึงมัลแวร์ที่เมื่อถูกติดตั้งแล้วจะทำการสร้างการเชื่อมต่อไปยังผู้ไม่ประสงค์ดี โดยอาจรอรับคำสั่งโดยยังไม่สร้างความเสียหายให้กับความมั่นคงปลอดภัยของระบบ แต่เมื่อได้รับคำสั่งอาจทำการอย่างอื่นๆเช่น คัดลอกข้อมูล ดักฟังข้อมูล หรือเป็นฐานสำหรับใช้โจมตีเครื่องอื่นๆที่เชื่อมต่อในเครือข่ายเดียวกัน เป็นต้น
\end{itemize}
ในปัจจุบันมัลแวร์ได้รับการพัฒนาให้มีความสามารถและประสิทธิภาพสูงขึ้นโดยสามารถสร้างความเสียหายและทำลายความมั่นคงปลอดภัยได้หลายหลายและยากต่อการตรวจจับมากยิ่งขึ้นเนื่องจากกาเผยแพร่ซอฟต์แวร์ต้นแบบสามารถกระทำได้ง่ายผ่านเครือข่ายอินเทอร์เน็ต
\subsection{การเจาะระบบ (Exploitation)}
การเจาะระบบสามารถกระทำได้หลากหลายวิธีขึ้นอยู่กับทักษะและช่องโหว่ที่เกี่ยวข้องดังนั้นรูปแบบการโจมตีที่เกิดขึ้นจึงขึ้นอยู่กับกระทบที่ผู้โจมตีต้องการอันได้แก่ การสกัดขัดขวาง การดักรับดักฟัง การเปลี่ยนแปลงแก้ไข และการปลอมแปลง ทรัพยากรสารสนเทศที่ตกเป็นเป้าหมายของการโจมตี ทั้งนี้เป้าหมายหลักจึงเป็นระบบและบริการต่างๆที่เชื่อมต่อกับเครือข่ายอินเทอร์เน็ตและเครือข่ายภายใน โดยผู้โจมตีจะเลือกพิจารณาโจมตีต่อช่อวโหว่หรือข้อบกพร่องด้านการรักษาความมั่นคงปลอดภัย โดยหากโจมตีต่อทรัพยากรหนึ่งสำเร็จก็มักจะใช้ทรัพยากรนั้นเป็นฐานสำหรับการโจมตีในลำดับถัดๆไป รูปแบบการโจมตีที่สำคัญมีดังต่อไปนี้

วิศวกรรมสังคม (Social Engineering) หมายถึง การโจมตีต่อทรัพยากรสารสนเทศผ่านความอ่อนแอในการตัดสินใจของมนุษย์ มีลักษระคล้ายคลึงกับการปฏิบัติการจิตวิทยา วิธีการโจมตีด้วยเทคนิคดนี้จะไม่มีความเกี่ยวข้องกับความรู้ความชำนาญเกี่ยวกับระบบสารสนเทศและการสื่อสารสูงแต่มุ่งเน้นในการสร้างสภาวะการณ์ที่เหมาะสมทำให้เหยื่อหลงเชื่อแล้วทำอย่างใดอย่างหนึ่งที่อาจทำให้การโจมตีด้วยเทคนิคอื่นๆประสบความสำเร็จ เช่น การหลอกให้บางคนหลงกลเพื่อเข้าถึงระบบด้วยการหลอกถามรหัสผ่าน การหลอกให้ส่งข้อมูลที่สำคัญให้ จะเห็นว่าการโจมตีด้วยเทคนิควิศวกรรมสังคมเป็นจุดอ่อนที่ป้องกันยากเพราะเกี่ยวกับคน แต่สามารถลดผลสำเร็จของการโจมตีได้ด้วยการกำหนดีขั้นตอนการปฏิบัติที่เป็นมาตรฐาน และการสร้างความตระหนักรู้ให้กับบุคคลากรที่เกี่ยวข้องกับระบบในทุกๆระดับตั้งแต่ผู้บริหาร ผู้ใช้งาน และผู้ดูแลระบบ

การเดารหัสผ่าน (Password Guessing, Bruteforce) หมายถึงการโจมตีด้วยการส่งข้อมูลการพิสูจน์ตัวจริงเข้าสู่ระบบด้วยข้อมูลกลุ่มตัวอักษรและเลขที่ใช้สำหรับการพิสูจน์ทราบตัวจริงของผู้ใช้ โดยปกติรหัสผ่านจะใช้คู่กับชื่อผู้ใช้หรือยูสเซอร์เนม (Username) สำหรับล็อคอินเข้าสู่ระบบซึ่งเป็นกลไกสำคัญในการพิสูจน์และกำหนดสิทธิ์ที่ผู้ใช้งานพึงมีต่อระบบ ดังนั้นการป้องกันการโจมตีด้วยเทคนิคนี้จึงต้องกำหนดและบังคับนโยบายที่เกี่ยวข้องกับวิธีการพิสูจน์ตัวจริงโดยห้ามใช้งาน
\begin{itemize}
	\item รหัสผ่านที่สั้น เช่น 123456, asdfgjkl เป็นต้น
	\item คำที่รู้จักและคุ้นเคย เช่น password, blue, admin เป็นต้น
	\item ใช้ข้อมูลส่วนตัวในรหัสผ่าน เช่น ชื่อ หมายเลขโทรศัพท์ วันเกิด เป็นต้น
	\item ใช้รหัสผ่านเดียวกันกับทุกๆ ระบบที่ใช้
	\item เขียนรหัสผ่านไว้บนแผ่นกระดาษแล้วเก็บไว้ในที่ ๆ หาได้ง่าย
	\item ใช้พาสเวิร์ดเดิมๆเป็นระยะเวลานาน
\end{itemize}


การโจมตีแบบปฏิเสธการให้บริการ (Denial of Service) หมายถึงการโจมตีที่ผู้โจมตีหวังผลให้ระบบที่ตกเป็นเป้าหมายไม่สามารถให้บริการได้ โดยผู้โจมตีอาจโจมตีต่อช่องโหว่หรือบั๊กของทรัพยากรสารสนเทศนั้นแล้วส่งผลให้ระบบปฏิเสธการให้บริการ หรือการโจมตีด้วยการร้องขอทรัพยากรพร้อมๆกันจำนวนมากจากเครือข่ายของซอฟต์แวร์อัตโนมัติที่เรียกว่า บอท (botnet) ซึ่งผลการโจมตีจากเทคนิคดังกล่าวจะทำให้ผู้ใช้งานที่มีสิทธิ์ไม่สามารถเข้าถึงและใช้งานทรัพยากรนั้นๆได้ และในบางกรณีอาจเปิดโอกาสให้ผู้ไม่ประสงค์ดีสามารถฝังซอฟต์แวร์มุ่งประสงค์ร้ายเช่น โทรจัน หรือคีย์ล๊อกเกอร์ ลงบนเป้าหมายเพื่อแฝงตัวเข้าโจมตีด้วยเทคนิคดอื่นๆต่อไป


การโจมีแบบคนกลาง (Man-in-Middle Attacks)
อีกรูปแบบหนึ่งของการโจมตีคือ การพยายามที่จะใช้บัญชีผู้ใช้ที่ถูกต้องในการล็อคอินเข้าไปในระบบ ซึ่งการให้ได้มาซึ่งข้อมูลเหล่านี้ก็โดยการใช้การโจมตีแบบคนกลาง การโจมตีแบบคนกลางของการสื่อสารผ่านระบบคอมพิวเตอร์ เป็นรูปแบบที่พบเห็นได้ทั่วไป การโจมตีประเภทนี้จะทำให้คอมพิวเตอร์สองเครื่องดูเหมือนว่าจะสื่อสารกันอยู่ โดยที่ไม่รู้ว่ามีคนกลางคอยเปลี่ยงแปลงข้อมูลอยู่การป้องกันการโจมตีแบบคนกลางก็อาจใช้วิธีการเข้ารหัสข้อมูลควบคู่กับการ พิสูจน์ทราบตัวจริงของคู่รับคู่ส่ง การโจมตีแบบคนกลางแบ่งออกเป็น 2 ประเภท คือ แบบแอ็คทีฟ (Active) และแบบพาสซีฟ (Passive) สำหรับแบบแอ็คทีฟนั้นข้อความที่ส่งถึงคนกลางจะถูกเปลี่ยนแปลงแล้วค่อยส่งต่อ ถึงผู้รับ ส่วนแบบพาสซีฟนั้นจะส่งต่อข้อความเดิมที่ได้รับ


\subsection{EMP Weapons}
EMP Weapons หมายถึงอาวุธที่ถูกออกแบบขึ้นเพื่อแพร่คลื่นอิเล็กโทรแมคเนติก โดยการส่งคลื่นแม่เหล็กไฟฟ้ากำลังสูงซึ่งส่งผลให้อุปกรณ์อิเล็กทรอนิกส์ได้รับความเสียหาย อย่างไรก็ดียังไม่มีรายงานผลสำเร็จของการใช้งานอาวุธลักษณะนี้ในการทำการรบ แต่ปัจจุบันมีการวิจัยเพื่อพัฒนาประสิทธิภาพและผลสำเร็จของการใช้อาวุธลักษณะนี้อย่างต่อเนื่อง

\section{เครื่องมือเชิงรับ (Defensive Tools)}
\subsection{การควบคุมการเข้าถึงทรัพยากร}
การควบคุมการเข้าถึงทรัพยากร
การควบคุมการเข้าถึงทรัพยากรสารสนเทศจะถูกดำเนินการโดยการกำหนดและใช้งานมาตรการควบคุม (access control) ซึ่งหมายถึง กระบวนการ วิธีการ หรือระบบซึ่งจะทำการตรวจสอบผู้ใช้งานก่อนอนุญาตให้ผู้ใช้งานที่ผ่านการตรวจสอบนั้นเข้าถึงทรัพยากรสารสนเทศใดๆได้ ยกตัวอย่าง มาตรการควบคุมสำหรับการเข้าใช้งานเครื่องคอมพิวเตอร์ส่วนบุคคลโดยทั่วไปที่นิยมใช้คือ การป้อนรหัสผู้ใช้งานและรหัสผ่าน มาตรการควบคุมสำหรับการผ่านเข้าออกห้องอาจมีการติดตั้งระบบคีย์การ์ดให้เฉพาะผู้ที่มีการ์ดเท่านั้นจึงจะสามารถเข้าออกได้ หรือแม้กระทั่งการแจกจ่ายกุญแจเฉพาะเจ้าหน้าที่ผู้มีหน้าที่เกี่ยวข้องในการเข้าถึงห้องใดห้องหนึ่ง ก็จัดเป็นมาตรการการควบคุม ทั้งนี้มาตรการควบคุมสามารถจำแนกเป็น 3 ประเภทคือ
\begin{itemize}
	\item การควบคุมการเข้าถึงภาพบังคับ (mandatory access control) หมายถึงหลักการควบคุมการเข้าถึงแบบที่ผู้ใช้งานไม่สามารถเปลี่ยนแปลงสิทธิการเข้าถึงทรัพยากรได้ด้วยตนเอง เหมาะสำหรับการควบคุมทรัพยาการที่มีข้อมูลและสิทธิ์ของผู้ใช้งานมีความชัดเจน เนื่องจากผู้ใช้งานแต่ละคนจะถูกแบ่งมอบสิทธิในการเข้าถึงทรัพยากรเป็นกลุ่มๆ (classes or categories) ในแต่ละกลุ่มจะมีการจัดลำดับความมั่นคงปลอดภัยเช่นการแบ่งข้อมูลออกแบบ ลับที่สุด ลับมาก ลับ และ ไม่จัดลำดับชั้นความลับเป็นต้น
	\item การควบคุมการเข้าถึงโดยผู้ใช้ (discretionary access control) มีหลักการการควบคุมการเข้าถึงทรัพยากรในลักษณะของการให้สิทธิแก่เจ้าของหรือผู้ได้รับสิทธินั้น เมื่อมีการร้องขอการพิสูจน์สิทธิ์จากผู้ใช้งาน กลไกการตรวจสอบสิทธิที่ได้รับอนุญาตของผู้ใช้งานจะถูกตรวจสอบ และกลไกนี้จะเป็นผู้ส่งต่อสิทธิที่ผู้ใช้งานได้รับให้สามารถเข้าถึงทรัพยากรได้อีกต่อหนึ่ง ซึ่งกลไกนี้เป็นกลไกมาตรฐานที่ระบบฐานข้อมูลนิยมใช้ในการควบคุมการเข้าถึง โดยทั่วไปเป็นที่เข้าใจได้ว่าผู้ใดสร้างหรือเป็นเจ้าของทรัพยากร ผู้นั้นจะสามารถเข้าถึงและมอบสิทธิการเข้าถึงให้แก่ผู้อื่นได้
	\item การควบคุมการเข้าถึงตามบทบาท (role-based access control) เป็นการควบคุมการเข้าถึงทรัพยากรตาม “หน้าที่” ที่ผู้ใช้งานมีต่อทรัพยากรสารสนเทศ กลไกควบคุมการเข้าถึงแบบนี้มีความเหมาะสมต่อการควบคุมการเข้าถึงทรัพยากรในระบบสารสนเทศ หรือโครงสร้างพื้นฐานระบบสารสนเทศที่มีความซับซ้อนเนื่องจากในระบบที่มีความซับซ้อนมากๆมักมีความต้องการควบคุมทรัพยากรที่หลากหลาย หน้าที่ของผู้ใช้งานจึงถูกนำมาพิจารณาในการกำหนดสิทธิ ทำให้มั่นใจได้ว่าจะไม่มีผู้ใช้งานคนใดที่สามารถเข้าถึงหรือบริหารระบบได้ทั้งหมด การที่มีผู้สามารถเข้าถึงหรือบริหารระบบได้แต่เพียงผู้เดียวย่อมมีความเสี่ยงในการที่ข้อมูลหรือสารสนเทศในระบบนั้นจะถูกเปลี่ยนแปลงแก้ไขอย่างไม่ถูกต้อง ตัวอย่างหนึ่งของการควบคุมการเข้าถึงตามบทบาทคือการควบคุมการเข้าถึงซอฟต์แวร์ที่ใช้ในการปรับแต่งคุณสมบัติของระบบปฏิบัติการซึ่งกำหนดให้ผู้ใช้งานต้องได้รับสิทธิ์เป็นผู้ดูแลระบบ (administrator) เป็นต้น
\end{itemize}


มาตรการการควบคุมที่ได้กล่าวมา มักถูกใช้ร่วมกับกลไกสำคัญต่อไปนี้ การแสดงตน การพิสูจน์ตัวจริง การกำหนดสิทธ์ และการกำหนดความรับผิดชอบ ในกรณีนี้จะยกตัวอย่างมาตรการควบคุมการเข้าถึงและใช้บริการธุรกรรมผ่านอินเทอร์เน็ตของสถาบันการเงินแห่งหนึ่ง ซึ่งลูกค้าสามารถเข้าใช้งานได้ผ่านเว็บบราวเซอร์ และซอฟต์แวร์ที่ทำงานบนโทรศัพท์เคลื่อนที่
\begin{itemize}
	\item การแสดงตน (identification) เป็นกลไกที่ใช้ในการควบคุมผู้ใช้งานที่ต้องการเข้าถึงทรัพยากรตามช่องทางที่ทรัพยากรนั้นๆกำหนดขึ้น ในกรณีนี้ผู้ใช้งานจะต้องใช้งานอาจถูกร้องขอให้ใช้งานผ่านเว็บบราวเซอร์ที่ได้รับความนิยมใช้งาน (เช่น ไฟร์ฟอกซ์ กูเกิ้ลโครม) และซอฟต์แวร์ที่ถูกพัฒนาขึ้นโดยสถาบันการเงินแห่งนั้นเท่านั้น หากการร้องขอใช้งานจากซอฟต์แวร์อื่นๆเช่น โอเปราบราวเซอร์ หรือซอฟต์แวร์ที่ไม่ได้ถูกพัฒนาขึ้นโดยสถาบันการเงินแห่งนั้น จะไม่เข้าถึงและพิสูจน์ตัวจริงได้
	\item การพิสูจน์ตัวจริง (authentication) เป็นกลไกที่ใช้ในการตรวจสอบความถูกต้องของผู้ที่มาแสดงตนขอเข้าถึงทรัพยากรสารสนเทศ การพิสูจน์ตัวจริงนิยมกระทำด้วยการตรวจสอบ “ความถูกต้อง” ของข้อมูลสำหรับการพิสูจน์ตัวจริง โดยแบ่งลักษณะของข้อมูลนั้นได้ 3 ลักษณะคือ
	\begin{itemize}
		\item ข้อมูลที่ผู้แสดงตนทราบ (something you know) เช่นชื่อผู้ใช้งาน รหัสผ่าน หมายเลขพินสำหรับใช้งานเอทีเอ็ม
		\item ข้อมูลที่ผู้แสดงตนมี (something you has) เช่นบัตรเอทีเอ็ม หมายเลขบัตร 
		\item ข้อมูลที่ผู้แสดงตนเป็น (something you are) เช่นข้อมูลลายนิ้วมือ ข้อมูลม่านตา 
	\end{itemize}
	สำหรับกรณีการเข้าถึงบริการธนาคารอิเล็กทรอนิกส์ในปัจจุบันนิยมใช้กลไกในการพิสูจน์ตัวจริงโดยใช้แหล่งที่มาของข้อมูลร่วมกันเพื่อให้มั่นใจได้ว่าผู้ที่แสดงตนนั้นเป็นผู้ที่มีสิทธิเข้าถึงทรัพยากรนั้นจริงๆ โดยการใช้ข้อมูลชื่อผู้ใช้งาน รหัสผ่าน ร่วมพาสเวิร์ดแบบใช้ครั้งเดียว (one-time password: OTP) ที่ระบบจะเป็นผู้ส่งรายละเอียดไปยังโทรศัพท์มือถือเป็นครั้งๆไป เป็นต้น
	\item การกำหนดสิทธิ์ (authorization) คือกลไกในการตรวจสอบและส่งมอบสิทธิ์สำหรับการเข้าถึงทรัพยากรสารสนเทศให้กับผู้ใช้งานที่ผ่านการพิสูจน์สิทธิตามที่ได้กำหนดไว้ในมาตรการการควบคุมการเข้าถึงสารสนเทศสำหรับผู้ใช้งานรายนั้นๆ ซึ่งการกำหนดสิทธิ์ก็จะสอดคล้องกับประเภทของการควบคุมการเข้าถึงดังที่ได้กล่าวมาแล้วในตอนต้น
	\item การกำหนดความรับผิดชอบ (accountability) เป็นกลไกที่ทำให้มั่นใจได้ว่าผู้ใช้งานที่เข้าใช้งาน ตลอดจนผู้ไม่ประสงค์ดีที่พยายามเข้าใช้งานจะสามารถถูกตรวจสอบและเป็นผู้รับผิดชอบผลของการกระทำที่มีต่อทรัพยากรสารสนเทศนั้นๆได้ วิธีการและเทคโนโลยีสำคัญที่ใช้ในการตรวจสอบและกำหนดความรับผิดชอบคือการจัดเก็บข้อมูลการจราจร การจัดเก็บประวัติการใช้งานหรือที่นิยมเรียกว่าล็อก (logs) ของทรัพยากรต่างๆที่มี 
\end{itemize}

\subsection{การกำหนดสภาวะของทรัพยากร}
การเฝ้าตรวจความมั่นคงปลอดภัยทรัพยากรสารสนเทศ และการตอบสนองต่อการโจมตีที่เกิดขึ้นจากภัยคุกคามรูปแบบต่างๆต่อทรัพยากรสารสนเทศอย่างเหมาะสม จะทำให้มั่นใจได้ว่าการโจมตีที่เกิดขึ้นจะถูกบริหารจัดการได้อย่างมีประสิทธิภาพและใช้ทรัพยากรที่เกี่ยวข้องได้แก่ งบประมาณ บุคลากร กระบวนการบริการจัดการและการประยุกต์ใช้เทคโนโลยีที่เหมาะสม เหตุการณ์ต่างๆที่เกิดขึ้นในทรัพยากรสารสนเทศจะถูกตรวจสอบและกำหนดสถานะให้กับทรัพยากรเหล่านั้นตามสภาวการณ์ที่แท้จริงโดยเป็นผลจากการบริหารความเสี่ยงด้านเทคโนโลยีสารสนเทศ ซึ่งทำให้ทราบว่าทรัพยากรใดมีความต้องการการเฝ้าตรวจอย่างไร ทั้งนี้สถานะของทรัพยากรๆนั้นจะถูกกำหนดเป็น 3 สถานะหลักๆได้แก่

\textbf{ภาวะปกติ (normal)} คือเหตุการณ์ปกติของทรัพยากรที่ถูกเฝ้าตรวจ จึงไม่มีความจำเป็นต้องถูกบริหารจัดการเนื่องจากผลการเฝ้าตรวจแสดงให้เห็นถึงสภาพปกติของทรัพยากรนั้นๆ

\textbf{ภาวะเฝ้าระวัง (escalation)} คือเหตุการณ์ที่เกิดขึ้นเมื่อผลการเฝ้าตรวจตรวจพบสิ่งสภาพผิดปกติของทรัพยากรสารสนเทศนั้นๆ โดยเหตุการณ์ดังกล่าวอาจส่งผลต่อความสามารถในการดำเนินการของทรัพยากร หรือการละเมิดมาตรการควบคุมบางอย่างที่ถูกกำหนดขึ้นไว้ในนโยบายการรักษาความมั่นคงปลอดภัย เมื่อเกิดเหตุการณ์ลักษณะนี้จึงมีความจำเป็นต้องได้รับการบริหารจัดการอย่างเหมาะสมจากผู้มีส่วนเกี่ยวข้อง

\textbf{ภาวะวิกฤต (emergency)} คือเหตุการณ์ที่เกิดขึ้นกับทรัพยากรสารสนเทศแล้วส่งผลกระทบต่อชีวิตและทรัพย์สินของมนุษย์ ผลกระทบจากภัยคุกคามต่อโครงสร้างพื้นฐานสำคัญของการดำเนินธุรกิจ ผลกระทบต่อเนื่องจากภัยคุกคามที่ทำให้เกิดสภาวะเฝ้าระวังบนทรัพยากรสารสนเทศแล้วส่งผลอื่นๆต่อทรัพยากรที่เกี่ยวข้องกัน หรือเหตุการณ์ที่เกิดขึ้นแล้วละเมิดนโยบายการรักษาความมั่นคงปลอดภัยอย่างร้ายแรง

\subsection{กระบวนการเฝ้าตรวจความมั่นคงปลอดภัย}
เมื่อเกิดเหตุการณ์ที่เกี่ยวข้องกับการรักษาความมั่นคงปลอดภัย ผู้รับผิดชอบความมั่นคงปลอดภัยของหน่วยจะเป็นผู้รับผิดชอบหลักในการแก้ไขสถานการณ์ให้กลับสู่สภาวะปกติโดยเร็ว ตามแผนเผชิญเหตุที่ถูกกำหนดไว้ล่วงหน้าเพื่อรองรับความเสี่ยงและภัยคุกคามที่อาจเกิดขึ้นร่วมกับคณะทำงานรับมือเหตุการณ์ด้านความมั่นคงปลอดภัย (Computer Security Incident Response Team: CIRT) ซึ่งถูกกำหนดหน้าที่ให้ตอบสนองต่อเหตุการณ์ด้านความมั่นคงปลอดภัย และให้บริการสิ่งจำเป็นสำหรับรับมือกับเหตุการณ์นั้นๆ เช่น การแจ้งเตือน การให้คำแนะนำ การอบรม และการบริหารจัดการเหตุการณ์ เป็นต้น โดยปกติกระบวนการบริหารความมั่นคงปลอดภัย
การบริหารความมั่นคงปลอดภัยประกอบการปฏิบัติหลักๆ ดังต่อไปนี้

\textbf{การเตรียมความพร้อม }(preparation) ในขั้นตอนนี้หน่วยจะทำการเตรียมความพร้อมสำหรับการเผชิญสถานการณ์ไม่พึงประสงค์ด้วยการฝึกฝนบุคลากร กำหนดนโยบาย ออกแบบกระบวนการ และเตรียมการทรัพยากรให้สอดคล้องกับเหตุการณ์ไม่พึงประสงค์ที่อาจเกิดขึ้น โดยอาจมีการเตรียมความรู้ ซอฟต์แวร์ และอุปกรณ์ที่จำเป็น เช่น ซอฟต์แวร์สำหรับการเฝ้าฟังเครือข่าย (sniffer) สายเคเบิลแบบครอส (crossover cable) แผ่นสำหรับการติดตั้งระบบปฏิบัติการ เป็นต้น ทั้งนี้การเตรียมความพร้อมอาจทำในรูปแบบของรายการที่หน่วย



\textbf{การเฝ้าระวังและการเฝ้าตรวจ }(detection and identification) เป็นขั้นตอนที่สำคัญอย่างยิ่งยวดเนื่องจากเป็นขั้นตอนเดียวที่จะสามารถบ่งชี้ได้ว่าเหตุการณ์ไม่พึงประสงค์นั้นเกิดขึ้นจริงๆ ในขั้นตอนนี้การวิเคราะห์อย่างรอบด้านจะทำให้ผู้มีหน้าที่รับผิดชอบและองค์กรสามารถดำเนินการตอบสนองต่อเหตุการณ์ไม่พึงประสงค์ได้อย่างเหมาะสม โดยหากการวิเคราะห์ผิดพลาดย่อมส่งผลเสียต่อการแก้ไขสถานการณ์เนื่องจากไม่สามารถแก้ไขปัญหาได้ตรงจุด หรือต้องใช้ระยะเวลาแก้ไขปัญหานานกว่าที่ควร

\textbf{การแก้ไขสถานการณ์วิกฤติ (incident response)} เป็นขั้นตอนที่คณะทำงานแก้ไขสถานการณ์เข้าแก้ไขเหตุการณ์ไม่พึงประสงค์เพื่อป้องกันการลุกลามของเหตุการณ์ไม่พึงประสงค์นั้นๆ การแก้ไขสถานการณ์อาจทำได้หลากหลายวิธีขึ้นอยู่กับสาเหตุของปัญหาที่แท้จริง ยกตัวอย่างเช่น หากเกิดเหตุการณ์โจมตีต่อระบบบริการสมาชิกเนื่องจากการโจมตีโดยการเปลี่ยนแปลงหน้าโฮมเพจ คณะทำงานแก้ไขสถานการณ์อาจตัดสินใจตัดการเชื่อมต่อของระบบบริการเว็บ แล้วสร้างช่องทางไปยังระบบสำรอง จากนั้นจึงทำการแก้ไขทรัพยากรของระบบเว็บเพื่อตรวจสอบผลกระทบของการโจมตีครั้งนั้นๆ 

\textbf{การบรรเทาสถานการณ์ (mitigation)} เป็นขั้นตอนที่เกี่ยวข้องกับการค้นหาผลกระทบที่เกิดขึ้นจากเหตุการณ์ไม่พึงประสงค์ เพื่อทำให้มั่นใจได้ว่าทรัพยากรที่ตกเป็นเป้าหมายนั้นถูกแก้ไข และสามารถฟื้นคืนสภาพกลับไปสู่การให้บริการได้ ดังนั้นการค้นหาสาเหตุที่จริงของเหตุการณ์ไม่พึงประสงค์จึงเป็นสิ่งสำคัญอย่างยิ่ง โดยทั่วไปคณะทำงานแก้ไขสถานการณ์จะทำการขจัดสาเหตุของเหตุการณ์ไม่พึงประสงค์ เช่นหากเหตุการณ์ไม่พึงประสงค์เกิดจากการฝังตัวของมัลแวร์ที่ทำการลักลอบดักรับข้อมูลการป้อนรหัสผ่าน คณะทำงานก็จะทำการตรวจสอบและกำจัดมัลแวร์นั้นๆและทำการป้องกันที่จำเป็นเพื่อป้องกันการเกิดเหตุการณ์ไม่พึงประสงค์นั้นๆซ้ำ

\textbf{การรายงานสถานการณ์ (reporting)} เป็นขั้นตอนที่กระทำควบคู่ไปกับการเฝ้าตรวจโดยการรายงานสถานการณ์จะต้องกระทำในโอกาสแรกตั้งแต่ผู้มีหน้าที่รับผิดชอบตรวจพบสิ่งผิดปกติขึ้น ทั้งนี้การรายงานจะต้องครอบคลุมทั้งข้อมูลเชิงเทคนิค และข้อมูลทั่วไปของทรัพยากรสารสนเทศที่ได้รับการเฝ้าตรวจ ข้อมูลทั่วไปของทรัพยากรสารสนเทศจะถูกใช้ในการประมาณสถานการณ์ของทรัพยากรในการดำเนินธุรกิจและธุรกรรมขององค์กร นอกจากนี้เมื่อเกิดเหตุการณ์ไม่พึงประสงค์ขึ้นผู้มีหน้าที่รับผิดชอบจะสามารถใช้ข้อมูลดังกล่าวในการประเมินผลกระทบต่อการดำเนินธุรกิจในภาพรวมและประมาณการณ์ผลเสียหายตลอดจนแนวทางการแก้ไขได้อย่างทันท่วงที

\textbf{การฟื้นคืนสภาพ (recovery)} เป็นส่วนหนึ่งของกระบวนการบรรเทาสถานการณ์และขั้นตอนที่ผู้มีหน้าที่รับผิดชอบทำการฟื้นคืนสภาพให้กับทรัพยากรที่ได้รับผลกระทบจากการโจมตี โดยปกติแล้วจะพิจารณาจากลำดับความเร่งด่วนของทรัพยากรนั้นๆต่อการดำเนินธุรกิจ โดยต้องคำนึงถึงความเป็นไปได้ที่ผู้มีหน้าที่รับผิดชอบอาจไม่สามารถกำจัดสาเหตุของเหตุการณ์ไม่พึงประสงค์ได้ ดังนั้นผู้มีหน้าที่รับผิดชอบจะต้องทำการเฝ้าตรวจและวิเคราะห์ทรัพยากรสารสนเทศนั้นๆอย่างใกล้ชิดเพื่อป้องกันไม่ให้เกิดเหตุการณ์ไม่พึงประสงค์ซ้ำขึ้น ด้วยการกำหนดตัวชี้วัดที่เป็นสิ่งบอกเหตุขึ้นแล้วทำการตรวจสอบตัวชี้วัดนั้นๆจนมั่นใจได้ว่าสาเหตุของเหตุการณ์ไม่พึงประสงค์นั้นๆถูกขจัดหมดสิ้นไปอย่างสมบูรณ์

\textbf{การฟื้นฟูสภาพ (remediation)} เป็นขั้นตอนหนึ่งที่กระทำอย่างต่อเนื่องโดยเป็นส่วนหนึ่งของกระบวนการบรรเทาสถานการณ์ โดยทำการวิเคราะห์และเฝ้าตรวจทรัพยากรในวงที่กว้างขึ้นกว่าจุดเกิดเหตุนั้นๆ ยกตัวอย่างเช่น องค์กรหนึ่งตรวจพบการแอบดักรับดักฟังข้อมูลบนเครือข่ายและคาดว่าผู้ไม่ประสงค์ดีสามารถดักรับดักฟังข้อมูลที่ใช้ในการพิสูจน์ตัวจริงต่อระบบบริหารทรัพยากรขององค์กร และค้นพบว่าปัญหานั้นเกิดจากผู้ไม่ประสงค์ดีติดตั้งมัลแวร์ลงบนคอมพิวเตอร์เครื่องหนึ่งของบริษัท คณะทำงานแก้ไขสถานการณ์จึงพิจารณาค้นหาและทำลายมัลแวร์ที่อาจถูกติดตั้งบนเครื่องคอมพิวเตอร์เครื่องอื่นๆที่เชื่อมต่อในเครือข่ายที่ถูกตรวจพบตั้งแต่เริ่มต้นพร้อมๆกับตัดการเชื่อมต่อเครือข่ายนั้นไปยังเครือข่ายอื่นๆในองค์กร และทำการบังคับนโยบายให้ผู้ใช้งานที่เกี่ยวข้องทำการเปลี่ยนพาสเวริ์ด เป็นต้น

\textbf{การถอดบทเรียน (lessons learned)} เป็นขั้นตอนที่สรุปรวมสิ่งที่เกิดขึ้นนับตั้งแต่เหตุการณ์ไม่พึงประสงค์อุบัติขึ้น โดยรวบรวมข้อมูลที่เกี่ยวข้องนับตั้งแต่การวิเคราะห์สาเหตุ การดำเนินการต่างๆที่เกี่ยวข้องโดยละเอียด โดยหวังผลให้ลดระยะเวลาตลอดจนการใช้งานทรัพยากรที่อาจถูกนำมาใช้เมื่อมีเหตุการณ์ไม่พึงประสงค์เกิดขึ้นครั้งๆถัดๆไป รวมถึงทำให้กระบวนการวิเคราะห์หาสาเหตุที่แท้จริงสามารถทำได้อย่างมีประสิทธิภาพมากยิ่งขึ้นเนื่องจากมีการสรุปรวมองค์ความรู้ต่างๆที่เกี่ยวข้องกับการแก้ไขสถานการณ์ไม่ถึงประสงค์


\end{document}