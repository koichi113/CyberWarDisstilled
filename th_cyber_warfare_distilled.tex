\documentclass[14pt,extrafontsizes]{extbook}
\usepackage{geometry}                % See geometry.pdf to learn the layout options. There are lots.
\geometry{a4paper}                   % ... or a4paper or a5paper or ... 
%\geometry{landscape}                % Activate for for rotated page geometry
%\usepackage[parfill]{parskip}    % Activate to begin paragraphs with an empty line rather than an indent
\usepackage{graphicx}
\usepackage{amssymb}
\usepackage{subfiles}
\usepackage{wrapfig}\usepackage{wrapfig}


\graphicspath{{images/}{../images/}}
\renewcommand{\contentsname}{สารบัญ}
\renewcommand{\chaptername}{บทที่}
\renewcommand{\figurename}{ภาพที่}
\renewcommand{\bibname}{บรรณานุกรม}

% Will Robertson's fontspec.sty can be used to simplify font choices.
% To experiment, open /Applications/Font Book to examine the fonts provided on Mac OS X,
% and change "Hoefler Text" to any of these choices.

\usepackage{fontspec,xltxtra,xunicode}
\defaultfontfeatures{Mapping=tex-text}
\setromanfont[Mapping=tex-text]{TH SarabunPSK}
\setsansfont[Scale=MatchLowercase,Mapping=tex-text]{TH SarabunPSK}
\setmonofont[Scale=MatchLowercase]{TH SarabunPSK}
\setmainfont{TH SarabunPSK}
\XeTeXlinebreaklocale 'th_TH'
\title{Cyber Warfare Distilled}
\author{น.ท.กรกช  วิไลลักษณ์ ร.น.}
\date{}                                           % Activate to display a given date or no date

\begin{document}
\maketitle
\tableofcontents


% For many users, the previous commands will be enough.
% If you want to directly input Unicode, add an Input Menu or Keyboard to the menu bar 
% using the International Panel in System Preferences.
% Unicode must be typeset using a font containing the appropriate characters.
% Remove the comment signs below for examples.

% \newfontfamily{\A}{Geeza Pro}
% \newfontfamily{\H}[Scale=0.9]{Lucida Grande}
%\newfontfamily{\J}[Scale=0.85]{Osaka}

% Here are some multilingual Unicode fonts: this is Arabic text: {\A السلام عليكم}, this is Hebrew: {\H שלום}, 
%and here's some Japanese: {\J 今日は}.


\subfile{chapters/01_cyber_threats}



\subfile{chapters/02_military_doctrine}



\subfile{chapters/03_cyberwar_doctrine}



\subfile{chapters/04_cyberwar_tools}



\subfile{chapters/05_cyberwar_tools}



%\subfile{chapters/06_cyberwar_the_future}



\XeTeX

\addcontentsline{toc}{chapter}{บรรณานุกรม}

\nocite{Libicki2009,Tobergte2013,Friedman2014,Andress2014,Subrahmanian2015}

\bibliographystyle{ieeetr}

\bibliography{bibliography}


\end{document}  